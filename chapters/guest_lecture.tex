\section{The Internet Computer}

This was a guest lecture by DFINITY. The internet computer is a platform to run any computation, using blockchain technology for decentralization and security.\medskip

The \textbf{internet computer protocol} (IPC) coordinates nodes in independent data centers, jointly performing computation for anyone. It create the internet computer blockchains. It guarantees safety and liveness of smart contract execution despite byzantine nodes. \medskip

A smart contract is called a \textbf{canister} and stores memory pages in data and code in the form of webassembly bytecode. Together they are a collection of replicated state machines. \medskip

Internet computer consensus is based on the assumption $n > 3f$ an guarantees consensus under asynchrony and termination under partial synchrony. \medskip

Each subnet has a network nervous system (NNS) as a backbone, performing administrative stuff, and a own public key. \medskip

Messages are placed in blocks to reach consensus. There are two types of messages, user to canister and canister to canister. Forming the blocks is done by a block maker. A random ranking of the block maker in each round is introduced by a shared coin. \medskip

\subsection{HTTP Outcall}

Since a smart contract should be able to execute every computation, we need to think about how to communicate with the off-chain world. This is done via a oracle service. The problem is that the oracles must be trusted. \medskip

The internet computer can communicate with no intermediary. Responses go through consensus to ensure deterministic behavior. 